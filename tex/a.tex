\documentclass[xhtml, mathjax]{article}

%\usepackage{fourier}
\usepackage{graphicx}
\usepackage{amsmath}
\graphicspath{{../doc/images}}

%\renewcommand{\thesubsection}{\thesection.\alph{subsection}}


\title{Visualizing Complex Functions}
\author{}
\date{}

\begin{document}
  \CssFile[custom.css]
  \EndCssFile
  \section{Introduction}

  I'll refer to functions that take complex numbers and return complex numbers
  as complex functions. These are hard to graph, since they essentially take two
  variables (the real part and the imaginary part of a complex number) and
  return two variables (the real part and the imaginary part of a complex
  number). So, the most straightforward way to graph them would be four
  dimensional.

  We can graph the real part and the imaginary part separately, but these are
  often related in subtle ways, as I'll show shortly.

  One common way these functions can be defined is called domain coloring. We
  write the complex number in polar form $(re^{i\theta} = r\cos(\theta) +
  ir\sin(\theta))$, and assign a color to every value of $\theta$. We then graph the
  complex plane, assigning every point $z$ the color $f(z)$ represents. The
  following image graphs $f(z) = z$, and we can see that $\theta = 0$ is is
  assigned a color halfway between blue and green, $\theta = \frac{\pi}{3}$ is
  blue, $\theta = \pi$ is red.

  \begin{center}
    \includegraphics{z_color.gif}
  \end{center}

  The reason it's a color wheel and not something like a gradient is because the
  colors can be the same value at $0$ and $2\pi$, so similar values of $\theta$
  are always similar colors.

  Also, any information about $r$ is not represented in this image, so sometimes
  contour lines are added. The image below colors a pixel black if $|f(z)|$ is
  close to a whole number, resulting in contour lines, similar to topographical
  maps. The gradient also goes from red to white from $-\pi$ to $\pi$ to illustrate
  the previous point.

  \begin{center}
    \includegraphics{z_contour.gif}
  \end{center}

  The novelty of this document is that instead of color, we represent each value
  of $\theta$ as a point in time in a moving image. In the image below, $\theta \approx 0$ is
  highlighted at first, and $\theta \approx \pi$ is highlighted halfway
  through. Similarly to the first approach, similar values of $\theta$ are
  treated similarly, since the difference between the last and the first frames
  is equivalent to the difference between the first and second frames.
  \begin{center}
    \includegraphics{z_loop.gif}
  \end{center}

  \section{Holomorphic Functions}

  Complex functions have a notion similar to differentiability in real numbers
  -- a complex function is said to be holomorphic if the following limit exists:

  \[f'(z) = \lim_{h \to 0} \frac{f(z+h) - f(z)}{h}\]

  Here, $h$ is a complex number, and the limit is defined as follows:

  \[\lim_{z \to a}f(z) = L \text{ means that for all } \epsilon > 0 \text{,
  there exists a } \delta \text{ such that } |z - a| < \delta \text{ implies }
  |f(z) - L| < \epsilon\]

  And $|z|$ here is defined as $|a+bi| = \sqrt{a^2+b^2}$.

  This is exactly the same way something like $\lim_{(x,y) \to (a,b)} f(x,y)$
  would be defined.

\end{document}

\EndPreamble
