\documentclass[xhtml, mathjax]{article}

%\usepackage{fourier}
\usepackage{graphicx}
\usepackage{amsmath}
\usepackage{hyperref}
\usepackage{subcaption}
\graphicspath{{../doc/images}}

%\renewcommand{\thesubsection}{\thesection.\alph{subsection}}


\title{Visualizing Complex Functions}
\author{}
\date{}

\begin{document}

  \CssFile[../tex/custom.css]
  \EndCssFile


  \section{About this Document}

    I initially came up with the idea for this document when I was reading Stein
    and Shakarchi's Complex Analysis in high school. I don't have the code I
    wrote back then, but there's an older version I wrote some time in between
    on my github. I haven't seen this idea anywhere else, but I wouldn't be
    surprised if someone else came up with it as well.

    The code for creating the images as well as the html page is on the github
    repo associated with this website,

    \url{https://github.com/yahya-tamur/complex-gifs}

    It should be easy to experiment and create your own images.

    Thanks for reading!

  \section{Introduction}

    I'll refer to functions that take complex numbers and return complex numbers
    as complex functions. These are hard to graph, since they essentially take
    two variables (the real part and the imaginary part of a complex number) and
    return two variables (the real part and the imaginary part of a complex
    number). So, the most straightforward way to graph them would be four
    dimensional.

    We can graph the real part and the imaginary part of the return value
    separately as functions of two variables, but these are often related in
    subtle ways.

    One common way these functions can be defined is called domain coloring. We
    write complex numbers in polar form $(re^{i\theta} = r\cos(\theta) +
    ir\sin(\theta))$, and assign a color to every value of $\theta$. We then
    graph the complex plane, assigning every point $(a,b)$ the color $f(a + bi)$
    represents. The following image graphs $f(z) = z$, and we can see that
    $\theta = 0$ is assigned red, $\theta
    = \frac{2\pi}{3}$ is green, $\theta = \pi$ is somewhere between blue and
    green.

    \begin{center}
      \includegraphics{z_color.gif}
    \end{center}

    The reason it's a color wheel and not something like a gradient is because
    the colors can be the same value at $0$ and $2\pi$, so similar values of
    $\theta$ are always similar colors.

    Also, no information about the $r$ in $re^{i\theta}$ is represented in this
    image, so sometimes contour lines are added. The image below colors a pixel
    black if $r$ is close to a whole number, resulting in contour lines,
    similar to topographical maps. The gradient also goes from red to white from
    $0$ to $2\pi$ to illustrate the previous point.

    \begin{center}
      \includegraphics{z_contour.gif}
    \end{center}

    The novelty of this document is that instead of color, we represent each
    value of $\theta$ as a point in time in a moving image. In the image below,
    $\theta \approx 0$ is highlighted at first, and $\theta \approx \pi$ is
    highlighted halfway through. Similarly to the first approach, similar values
    of $\theta$ are treated similarly, since the difference between the last and
    the first frames is equivalent to the difference between the first and
    second frames.

    \begin{center}
      \includegraphics{z_loop.gif}
    \end{center}

    It can be hard to tell which direction $|f(z)|$ increases by looking only at
    the contour lines. In the image below, the contour lines move from lower to
    higher values.


    \begin{center}
      \includegraphics{z_contour_loop.gif}
    \end{center}

    While graphing $\frac{1}{z}$, notice that$\frac{1}{re^{\theta i}} =
    \frac{1}{r}e^{-\theta i}$, so in addition to the contour lines moving in
    instead of out, the selected pixels moves clockwise instead of
    counterclockwise:

    \begin{figure}
      \centering
      \includegraphics{inv.gif}\par
      $f(z) = \frac{1}{z}$
    \end{figure}

    This process does create very interesting looking Moire patterns.


  \section{Holomorphic Functions}

    I might write some actual introduction to holomorphic functions here later,
    but for now, I just decided to put in the images with a few comments.

    Consider the linear function $a(z-b)$. $z-b$ is exactly the
    function we've seen before except the origin moved to $b$, and since
    $r_1e^{i\theta_1} \times r_2e^{i\theta_2} = (r_1 r_2) e^{i(\theta_1 +
    \theta_2)}$, multiplying by $a$ has the effect of the rotation starting at a
    different angle, and the contour lines having a different spacing:

    \begin{figure}
      \centering
      \begin{subfigure}{0.5\linewidth}
        \centering
        \includegraphics{linear_1.gif}\par
        $f(z)=5i(z-2)$
      \end{subfigure}
      \begin{subfigure}{0.5\linewidth}
        \centering
        \includegraphics{linear_2.gif}\par
        $f(z)=0.5(z-(3+2i))$
      \end{subfigure}
    \end{figure}

    Since two unique values of $\theta$, on opposite sides of the unit circle,
    solve $z^2 = C$, here's what $f(z) = z^2$ looks like:

    \begin{figure}
      \centering
      \includegraphics{poly_1.gif}\par
      $f(z) = z^2$
    \end{figure}

    And here's what it looks like there are two roots next to each other instead
    of a single double root:

    \begin{figure}
      \centering
      \includegraphics{poly_2.gif}\par
      $f(z) = z^2 + 1$
    \end{figure}

    And here's a seven-root:

    \begin{figure}
      \centering
      \includegraphics{poly_3.gif}\par
      $f(z) = z^7$
    \end{figure}


    Every holomorphic function is equal to a power series, however this doesn't
    mean they always look like a countable set of roots. Below, we have
    increasing partial sums for $e^z = \sum_{n=1}^\infty \frac{z^n}{n!}$,
    graphed on the same scale.

    \begin{figure}
      \centering
      \includegraphics{exp_sum_to_5.gif}\par
      $f(z) = \sum_{n=0}^4 \frac{z^n}{n!}$
    \end{figure}

    \begin{figure}
      \centering
      \includegraphics{exp_sum_to_10.gif}\par
      $f(z) = \sum_{n=0}^9x \frac{z^n}{n!}$
    \end{figure}

    \begin{figure}
      \centering
      \includegraphics{exp_sum_to_20.gif}\par
      $f(z) = \sum_{n=0}^{19} \frac{z^n}{n!}$
    \end{figure}

    \begin{figure}
      \centering
      \includegraphics{exp.gif}\par
      $f(z) = e^z = \sum_{n=0}^\infty \frac{z^n}{n!}$
    \end{figure}

  And if you think about it, the power series doesn't converge uniformly, so
  every partial sum will be closer to the limit in the center and pretty far
  from it towards the outside.

  Here's the inverse stereographic projection of the complex plane,
  onto a sphere. Notice that the function $f(z) = z$ looks like $f(z) =
  \frac{1}{z}$ on the opposite pole, which represents a point at infinity.

    \begin{figure}
      \centering
      \includegraphics{z_sphere.gif}\par
      $f(z) = z$
    \end{figure}

  \end{document}

\EndPreamble
